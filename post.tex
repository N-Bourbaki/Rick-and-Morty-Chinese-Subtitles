\documentclass{ctexart}

\usepackage{hyperref}
\usepackage{xcolor}

\renewcommand{\labelitemii}{\(\circ\)}
\renewcommand{\labelitemiii}{\textperiodcentered}

\begin{document}

\appendix\em

\section{原载于SubHD的招募公告}

{\color{red} 如果您阅读过这个公告,只想了解哪里有重要变化,请直接下拉至最后,那里集中罗列了此类信息,并适当附上了fjn308对这些改动的解释。}

公告如下:

如果你有兴趣加入我们的译制工作,或想了解“会不会在注释里解释梗”这类问题,或只是闲着没事儿,请阅读以下问答——

\subsection{你们制作什么?}

ass字幕文件,并且完全开源成品(SubHD、GitHub)、在发布后吸纳合理的修改建议(GitHub的pull request和merge)。

\subsection{会压制熟肉吗?}

不碰熟肉,也不对任何使用我们字幕作品生成的视频负责。

\subsection{字幕是双语的吗?匹配什么片源?是否包含特效、注释?}

一定是双语字幕,只匹配当下质量最高的片源,含特效(但做成简陋的、注释风格的样子,下详)、不含注释(即对梗们起解释作用的顶端文字)。

更具体地说:目前质量最高的片源是1080p.AMZN-CtrlHD.mkv,未来是蓝光;特效是指“必须的”特效,即视频画面中出现的、看不懂它就看不懂后续剧情的英文内容;不含注释的理由是——

就凭几个人的力量,梗是很难找全的。而且,梗给人的快感在于你发现它,如果一出现就立刻用注释告诉你,注释就成了电影院里恼人的后排美女的碎嘴男友。还有就是,梗经常不是一两句就能说得清的,不应该把一个需要用五句话才能说清楚的趣事儿缩成一句冰冷的概括让它一秒闪过。

\subsection{不含注释,那么观众要白白错过各种梗吗?}

当然不是!Chrim计划在微信公众号Community-137发表文章,其中不光会解释翻译、校对小伙伴在译制过程中发现的梗,我们还邀请所有人在回复中贡献新梗,丰富所有人的理解。

\subsection{我可以加入译制工作吗?可以参与改错吗?又是SubHD又是GitHub,你们究竟要怎么用这些平台?}

当然是两个当然!以发布字幕作品为节点,发布之前我们会在内部使用文档协作平台编辑字幕,为的是提高效率;得到成品后,字幕会被同时发布在SubHD和GitHub(如果只是下个字幕,你只需访问SubHD就够了);此后,所有完善工作都会被转移到GitHub进行,为的是更好的版本管理。

关于使用GitHub公开改错的细节,请前往\ref{gh}查阅。

\subsection{我感觉GitHub仿佛挺鸡肋的,要它做什么?}

第一,GitHub代表了开源精神,这是我们想声明的原则。

第二,在译制过程中,尽管有互相修改的工序,但每个人集中编辑自己的那块才是主要工作,此时版本控制不如高效率地集中同时编辑重要;在发布之后,如果字幕质量不过硬(我们并非专业组织),就会出现热心网友帮助校对的情况——这是我们希望的,但随后会出现的问题就是:

发布后我们也会不停修改,会去看热心网友改了什么地方,但不太可能不加取舍全部吸纳,还会再修改他也没提到的问题,然后得到了第二版——随后,热心网友也觉得自己的校对需要完善,那么他该修改哪个版本呢?再后来,我们这群强迫症又觉得一些地方需要改(但其实根本不怎么重要),一分钟改一个版本重新上传……再再后来,你……也……想……参与……修改……我觉得你已经理解了我所表达的问题。

因此,每个人都在SubHD开新页面发布校对版可能会好心办坏事儿,因此我们选择GitHub,一个在线的版本控制工具。热心网友可以就自己的每处建议发送修改请求(pull request),合理的那些会被我们合并(merge)到字幕中,这就是GitHub的用处,而SubHD只会发布阶段性稳定的成品。

\subsection{我想加入你们,需要怎么做?}

请添加Chrim的微信(cream\_scream),他会拉你入群,目前就用这个简陋途径。我们非常需要热心的译者加入,如果你有能力,请不要再犹豫,赶紧微信去加cream\_scream。

\subsection{公告的更新}

\subsubsection{S04E04}\label{gh}

\begin{itemize}
    \item 首先,我们给自己起了个名字:Community-137
    \item 其次,申请了以这个名字为ID的微信公众号。公众号和下载字幕永远不会有任何绑定关系,不会搞什么脑残流量变现,更不会像傻逼一样把大家折腾去扫码啥的,只是解读梗的文章需要一个地方发布
    \item 最后,关于GitHub公开改错,请前往fjn308在GitHub的主页: 
    
    \href{https://github.com/fjn308?tab=repositories}{\texttt{https://github.com/fjn308?tab=repositories}},
    
    点开任意一个以\texttt{rm}开头的项目,即可看到详细说明
\end{itemize}
\end{document}
