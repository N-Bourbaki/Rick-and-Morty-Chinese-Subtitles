\documentclass{ctexart}

\title{Community-137:字幕制作手册}
\author{fjn308}
\date{v1.12}

\usepackage{hyperref}
\usepackage{ulem}
\usepackage{xcolor}

\renewcommand{\labelitemii}{\(\circ\)}
\renewcommand{\labelitemiii}{\textperiodcentered}

\begin{document}

\maketitle

\section{粗略描述}

收到文档别怕长,因为\emph{\small 楷书小字}部分一般可以跳过:这些文字有些是不重要的细枝末节,有些是则用于解释某个规则为什么被设计成这样——换句话说,如果你觉得没必要深究,或未感到有不合理的说法即可跳过,否则,对于后一种情况,\emph{\small 楷书小字}是fjn308试图用来说服你的理由,但如果你\uwave{经过一番思考}仍不认可,请立即与fjn308讨论。

这一节为你建立对工作流程的初步印象。没有在本节中具体描述的,要么是后文中会被细致说明的重点,要么就是fjn308打杂的内容,从略。

\subsection{片源}

周一下午一点前后,TV版片源释出,有消音,文件中封装了字幕,其中包含[bleep]等字样;最早半夜12:00左右,最晚无穷远,CtrlHD发布未消音的AMZN版。AMZN版内封字幕轨的时间轴不同于TV版,但断句位置基本相同、{\color{red} 文本基本一致,就连错误也一致,有时候甚至时间轴还有问题——唯一优势在于[bleep]被替换为了正确的fuck、shit等}。

简而言之,TV版视频的唯一优势就是出得早,{\color{red} AMZN版字幕的唯一优势是有脏话}。

\subsection{翻译、一校;二校、改错}\label{correction}

TV版的时间优势让它成为我们翻译的蓝本。翻译过程中无需把[bleep]替换为fuck——再说一遍,千万别费这劲!何况咱也听不准啊!{\color{red} 但是,随着对AMZN版字幕来源的了解更多——其实就是把TV版的改了[bleep]调了轴还经常调错,此处的方案变化为:在翻译中如果遇到明确的英字错误,请修改之}。至于调轴和修正[bleep],是fjn308的打杂内容,大家忽略即可。

翻译之后,参与翻译的这几个人相互校对一次,这就是所谓一校,完成后这几个人就可以休息了。字幕发布之前,fjn308还要拉来一位高手一起进行二校,这将是被发布到SubHD和GitHub的第一版字幕。此处高手目前是se9fault,这个安排巧妙地利用了时差。

第二天开始,我们内部通过讨论为第一版字幕改错,{\color{red} 其实就是在群里唠嗑,不很正式,唠多唠少取决于有多少可唠的},fjn308会按照大家讨论的结果去GitHub陆陆续续修修改改;另一方面,在外部,GitHub上的网友也会提交pull request帮助改错,fjn308也会merge合理的改动。本段所述组织改错、向GitHub提交修改的工作,fjn308有意在S04E06或更晚起转交给衣柜大佬季叽馥疾己来做。

\subsection{特效、注释}

这两件事可能是值得说说的最大的小事儿了,故在这里阐述对待此二位的原则。本文对特效的定义是:画面中的文字;对注释的定义是:对译文或剧情的注解,多用来展开梗。关于它们,我们的做法是:

\subsubsection{特效}

特效从简,即只翻译“如果不译出会使观众对剧情的表层理解产生障碍”的画面内容。\emph{\small 例如,S04E03中的盗窃展Heist Con就属于这种,因为随后Rick进门时说“不如叫用力过猛展”;但S04E01中一群虚拟Rick反对Morty不克隆自己时举的标语牌就不算,这是看懂了更好但不认识也没啥影响的内容,不属于妨碍“表层理解”的范畴。}

另外,我们只以最朴素的顶端注释的形式给出上述译文——严格讲它们够不上“特效”这个词,只能叫伪·特效。\emph{\small 早期版本的文件中,fjn308曾建议把真·特效这种艺术形式在后续发布的sup格式字幕中发扬光大,但这部分讨论已经被他删除,发布sup文件的计划也随之取消。原因是对特效艺术的追求已经偏离了我们最根本的方向,意义感缺失。对此,fjn308认为不必再浪费篇幅赘述之。}

\subsection{注释}

不写注释,更确切地说,就是不对梗做出解释。\emph{\small 这个看起来有病的想法来源于以下几个观点:1,梗不是一句冰冷的概括就能说得清的;2,被迫的缩句是种屈辱,特别是一句话概括一部电影或电视剧,且不说这也不能给没看过的人带来多大信息量;3,赶在观众get到点之前就把梗说出来,多妙的梗都会瞬间变得无趣;4,这爷俩语速够快了,观众根本看不过来;5,就凭咱们几个找,很难把梗快速找全,更新也是个问题。}

{\color{red} 此处另外值得强调的是:译文也要尽可能忠实原文,原则上避免引入文本之外的解读成份,或者说避免掺入我们自己的“小聪明”。最好的平衡是:外国人听了英文能联想到的暗示,中国人看过中文也能联想到——但不由我们直白给出。}

对失去注释的补救,就是以公众号(ID: Community-137)文章的形式解读梗或剧情。这来源于Chrim的提议,具体怎么执行也由Chrim把握。\emph{\small 由公众号联想到的另一个问题是随着群规模的变大,可能有一天会分化出粉丝群和工作群,如果有那么一天,fjn308希望黄金矿工走吃元愿意管理粉丝群。}

\section{工作详述}

这一节不按照时间顺序描述所有事情,而是优先详细叙述最重要的环节:翻译和一校。下面先说一些背景。

参与翻译的朋友拿到手的将是一个,石墨文档上,即网页上,显示的带[bleep]字样的文本文档,示例其中一句如下:

{\small
\begin{verbatim}
00:00:01,691 --> 00:00:04,038
your_name翻译
Rick, I-I'm not rated to climb something this sheer.
\end{verbatim}
}

这里的\texttt{your\_name}是分工标记,这句交给谁翻译\texttt{your\_name}就变成谁的名字。以上内容记作A,记经过翻译、一校得到的是B:

{\small
\begin{verbatim}
00:00:01,691 --> 00:00:04,038
Rick 这么陡的坡我爬不了
Rick, I-I'm not rated to climb something this sheer.
\end{verbatim}
}

在这里提出翻译的第零条规则是合适的:\uwave{只把“\texttt{your\_name翻译}”逐个替换成译文即可,不要去掉它形成空行,也不要改动其他任何地方。}\emph{\small 这样设计的原因完全在于srt文件的语法,这里不赘述。总之要达到的目的是,任何人都可以随时全选复制全部文本内容到本地,立即生成一个没有语法错误的srt文件供预览使用。}

而凌晨要发布的则是C:

{\small
\begin{verbatim}
Dialogue: 0,0:00:01.69,0:00:04.03,Chn,,0,0,0,,Rick 这么陡的坡我爬不了
\N{\rEng}Rick, I-I'm not rated to climb something this sheer.
\end{verbatim}
}

从B到C的工作属于到C的工作属于二校和fjn308打杂的范围,从略。\emph{\small 关于ass文件应该调用什么字体,等等这些关于样式选取的问题,在建群初期有过一些争论。为此fjn308曾用一个小节的篇幅展示了他认为合理的ass文件规范,并解释了为什么他认为过多讨论字体选取问题的意义不大,朴素待之即可。但这部分讨论现已过时,故移除之。}

这一节将要描述的是:

\begin{itemize}
    \item 翻译和一校,即从A到B的过程中,应当遵循的文本规范、文风规范
    \item 人员分配的基本规则
    \item 译校流程的方案框架
\end{itemize}

\subsection{文本规范}

以下几乎是约定俗成的共识,务必严格遵守,如有遗漏请帮忙补充:

\begin{itemize}
    \item 以空格为分隔符,不使用顿号、逗号、句号、问号、叹号、破折号和省略号等标点符号;可使用的标点符号仅限于:“中文双引号”、《中文书名号》以及*星号引用歌词*。\emph{\small 对此,亦有很多字幕组的做法是使用``英文双引号''甚至\texttt{"}直引号\texttt{"}、<中英文都不这么用的大于号小于号>,等等,私以为这种扭曲的用法大可不必,把多数标点变成空格避免闹眼睛就可以了。至于问号和叹号以及表示话没说完的破折号、省略号为什么不留:语气自明,大可省略}
    \item 倾向忽略行首或行尾没有必要译出的拟声词(Fuck这类有单词的不算拟声词),比如\sout{Ooooweee}
    \item 用“-”标记同一行中的内容分属不同说话者,如:- 什么冬梅 - 马冬梅啊,注意把“-”与前后内容用空格隔开。\emph{\small 在石墨文档中,你会遇到在行首键入的横线被自动替换成圆点的情况。推荐的解决方法是先打一个斜线或随便什么非空字符,把这一行保护起来,再正常输入,别忘了最后把那个多余的东西去掉}
    \item 没有含义的名字、非著名人物的名字一律不译,将(疑似)有含义但没有标准译法(典型的有标准译法的名字是地名和名人的名字)的名字\colorbox{green}{标绿},留待集中讨论译名
\end{itemize}

\subsection{文风规范}\label{trans}

这部分内容是新的,其中规定之内容相当主观。这部分存在的意义仅在于统一译文各部分的文风,可能会持续扩充,如下:

\begin{itemize}
    \item God、Jesus这类词汇,用于感叹时,特别是出自Rick之口的,优先译作“天”。\emph{\small fjn308建议尽量避免使用“上帝”、“耶稣”的说法,“神”还行,但也差一点。剧中大概没有就Rick对宗教的看法做过明确的讨论,即便有,我们也应尽量避开这个问题。对于物理学家,这件事经常并不随意。fjn308对(百年之内的或在世的,牛顿时代的不算)物理学家的了解是,他们一般是不可知论者(可能带有一些宗教情结,如杨振宁)或无神论者(可能激烈地反对宗教,如温伯格)}
    \item 避免生硬地字字对译的同时,更要避免过度意译——这是要在忠实原文和适应中文表达方式之间做出的合理折中,切忌过度发挥掩盖原意
\end{itemize}

\subsection{人员分配}

\em\small

在陈述我们的安排之前,先描述电波使用的方案:

\begin{itemize}
    \item 存在一名“总监”,他将英字分段后邮件群发,成员回复认领
    \item 成员邮件传回译本,完全服从总监的修改、润色——也就是回完邮件就结束参与
\end{itemize}

必须承认我们中目前没人能有自信跟不会犯错的电波总监PK,平均英文水平我们也不及电波,以上方案不能用。fjn308认为,类似总监的工作必须由二或多人共同完成;必须增加内部互校环节,来克服水平不够的问题——这两点是所有具体安排的出发点。

\em\normalsize

我们的工作模式还在探索阶段,因此没有过于固定的方案,但基本的构造是清楚的:5-13人分段翻译,每3-5段构成一组,分组一校,每组配备一位负责人,二校需要至少两人参与,至少是一位打杂和一位高手,发布后的改错也最好由一位高手带领。除此之外还有别的杂事儿,这里就不一一列举了。请大家积极自荐合适角色,并尽可能适应fjn308每次协调各位声音后制定的安排。\emph{\small 对此他把解释写在了下面:}

\em\small

值得花些面积说明的是,在早期的文档中,fjn308曾提出“元老院”规则,即定期投票选出三个人共同承担总监职责,这个过于理想主义却不适合实际操作的构想已被正式移除。目前我们的人员构成并不稳定,预期总会有高手涌入,一共就那么几集,投来投去不如他们一出现就果断启用;每个人的工作时间不确定,不可避免的参与度不均匀导致每个人手中的票的说服力参差不齐,票选的形式美观远盖过其实际意义。在此fjn308希望大家相信他的眼光:他会在各位报名的基础上指派合适的人做合适的事,也可能会轻微“逼迫”一些人去尝试他认为此人有潜力做好的事儿,当然也不会不顾对方反抗——fjn308认为这个程度的“独断”暂时有助于Community-137的运行。

\em\normalsize

请注意,在每次发出的文档中,下一小节中描述的方案框架都可能会发生变化,但一般不会出现连续两次之间的剧烈变化。

\subsection{方案框架}

\paragraph{专人打杂}

专人解决纯粹的翻译、校对工作以外的所有杂事,也就是保证多数人专注于译校而不必去管其他非中英文语言层面的琐碎技术细节。目前打杂工作交给fjn308,细碎而无聊,他有意培养新人,但还怕别人不够细致。

\paragraph{分段翻译,$\frac{24}{n}$小时}

具体怎么报名、分配十分随意,每次会在群里说清规则,不另赘述。重点:从这开始要有时间观念。一点前后出片源,考虑各种不稳定因素,认为两点能得到视频文件,如果下载出现障碍请向fjn308求助。从两点算起,啃生肉+翻译会有一个时限,一般取24除以分段数为小时数。打杂的下文件、抽字幕、预处理、分段落再传石墨需要半小时左右,因此刚下好片源的朋友请先轻松观影。建议的流程是:完整看一遍+仔细看一遍自己的那部分+翻译+仔细看一遍自己的那部分+修改+完整看一遍+修改。至于挑出画面中的特效,也是打杂内容,各位忽略即可。

\paragraph{一校:分组互校,1小时}

报名后分配段落序号的同时,会给出一个分组,一般是三人一组,还会为每组安排一个负责人。分段翻译完成后的一个小时里,组内循环互校,也就是每个人校对上一个人的,第一个人校对最后一个人的。校对的原则是仅仅修正语义错误,忽略表达习惯差异。最后fjn308还会稍作必要修改来统一语言风格。互校过程中,如果发现译文有语义不妥之处,请把相应的翻译\colorbox{yellow}{标黄},并在原译文后另起一行附上自己的翻译或使用石墨的评注功能对这一行发表修改意见。循环互校完成后,每个人回头去看自己被修改或建议了什么,如果认为言之有理,直接改正、去掉标黄即可。如果认为自己更有道理、对方理解有误,就向对方解释自己的道理。如果他认为你说的有理,或者其他形式地你们达到了共识,就确定译文,去掉标黄。

\paragraph{一校:解决分歧,1小时}

经过上述交流,如果双方仍有分歧或这俩人都觉得其实谁都没理解上去,\colorbox{red}{标红},找来所在分组的负责人讨论。这部分牵涉到三个人的讨论,限时也是一个小时。若得到一致结果,去掉标红,否则保留。三方都用石墨的评注功能尽可能多地留下值得参考的信息,供二校的人判断,至此一校结束。这里值得说明,每组的负责人可能是参与分段翻译的人,也可能不是,视人手富裕程度而定。对于后者的情况,负责人应该不断刷生肉尽力理解全片,为更好地解决分歧做好准备。同时,他可以随时从石墨全选复制文本到本地,生成临时的srt字幕辅助刷片。

\paragraph{二校、发布}

一校完成后,大多数朋友就可以去休息了。第二天早晨之前,fjn308会拉来至少一位高手共同完成二校,再做后处理、发布。不出意外,第二天早八点前第一版字幕就会上线,匹配的版本取决于AMZN片源是否已经释出。后续的改错(和可能的洗版)方案已在\ref{correction}最后一段提及,略。

\subsection{关于工作时间的补充说明}
yixia
如果你想要参与制作,请仔细、完整地阅读下面的内容,即便是哪句话让你感到自己不符合要求,也请继续阅读下去,因为下面先描述了“原则上”的要求,后描述了“苟且的”要求。fjn308需要明确的是:

原则上,除非另有指派,所有参与者都会被分配一段翻译任务。因此,从下午一点起,参与者需要能够连续在线工作。考虑到5-13人参与翻译,这个过程耗时$\frac{24}{n}$小时,即2-5小时,再加上随后两小时的一校,意味着一般需要连续工作4-7小时。另外,第二天的白天我们会在群里适当讨论,虽然不必全程在线,但至少要能做到有时间参与自己能做出实质贡献的那几句讨论,不浪费自己在前一天翻译工作中所获得的见解。

但是,只要你有fjn308看重的才华,却无奈时间限制了你施展拳脚,以上原则都可以打破。这种情况下,请理解联系fjn308,他会为你另作安排。

即便你自认为只是最最普通的参与者,怕出头,不愿意fjn308为你另作安排,而被时间限制的你仍真诚希望出力,也不要气馁,连续工作4-7小时仍然只是原则上的要求。大家都有自己的主业,搞得这么严格可能就没有几个人能参与了:比如在这几个小时里,五六点了你饿了,想吃个饭去,这类事都没问题的,我们不是纳粹。对于此类灵活性,目前的基本原则是:

\begin{itemize}
    \item 至少保证能从一点起连续在线,直到翻译完自己的那段内容;或采取更弱的要求:不论从几点开始在线,必须保证在一校正式开始时,能够完成自己的翻译任务。{\color{red} 此苟且原则对小组负责人不完全适用,适用程度请自行把握}。\emph{\small 这意味着我们原则上拒绝参与者要求在预设的翻译阶段过后突然介入,使得一校被整体延后,或使一校从一个不完整的翻译开始。对此fjn308有形象的比喻:不可以要求在所有人都扫完地后,特殊为某个迟到者以献爱心的精神保留他的份额,站着等他赶到,行注目礼看他扫完自己的那块地,然后再一起拖地,或在拖地时绕着那块没扫的地托——这是一段夸张的描述,表达了fjn308对此类行为的愤怒。原因不言自明,上个不明的人已被驱逐}
    \item 由此衍生的一个问题就是:参与翻译者,在难以逾越的时间障碍下,可以不参与一校,{\color{red} 小组负责人除外},其一校份额由其他同组成员分担——具体如何分担由该组负责人把握
    \item 衍生的另一个原则是:由于自身过于优秀而在预设的一校时间点到来之前就完成了翻译任务的参与者,应该压住自己的校对热情,利用这段时间打磨自己的译文(原则见\ref{trans}文风规范的最后一条),给同组其他成员留出专心空间,并且把自己的优秀不吝告知fjn308——他会在下次给你更适合自己的工作或工作量。{\color{red} 当然,小组负责人亦有权限平衡内部成员的意见,修改时间节点,或做出其他合理安排}
    \item 最终,也是最重要的原则:所有时间上的突发难处都不是问题——你唯一要做的,就是把这些难处第一时间告知fjn308
\end{itemize}

文末,感谢大家的热情。
\end{document}
